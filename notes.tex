\section{Atomic Structure and Behavior}
Atoms\index{atom} are considered the smallest units of ordinary matter with the
properties of a chemical element and are defined in terms of their constituing
parts - neutrons, protons and electrons. Atoms, typically sized around
\SI{100}{\pico\meter}, are so small that classical physics fails to accurately
describe their behavior due to quantum effects.

\section{Atomic Mass and Weight}
The colloquial usage of the term ``weight'' is technically incorrect as the
intent is actually to use the term ``mass''.

In physics mass is measured in grams (\si{\gram}) while weight is measured in
Newtons (\si{\newton}).
Weight\index{weight} $W = m\cdot g$ represents the force on a mass due to
gravity, as such weight is very dependent on the gravitational pull that a
mass is subjected to.
The accelleration of an object due to its subjection to a gravitational pull is
referred to as gravity\index{gravity} $g = a_{gravity} = \frac{v}{t} = \frac{d}{t^2}$
which, on the surface of Earth, results to an earthbound acceleration of about
\SI{9.8}{\meter\second^{-2}}.

In chemistry mass is expressed in atomic mass units (\si{\atomicmassunit})
while weight is referred to as atomic weight.

Atomic mass units\index{atomic mass unit} have been standardized to the new
unit \si{\atomicmassunit} in 1961 to refer to the mass of \ce{^{12}}C in order
to address the complication and confusion caused by chemists and physicists
upholding different standards
\footnote{Within Chemistry naturally ocurring oxygen (including the heavier
\ce{^{17}}O and \ce{^{18}}O) were considered as a reference for 16 atomic mass
units whereas in Physics the pure isotopic \ce{^{16}}O was considered the
reference for the same 16 atomic mass units, accounting for a difference of
\num{1.000282} at times between the two fields.} for atomic mass units (amu)
prior.
