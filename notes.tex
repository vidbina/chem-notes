\section{Atom Fundamentals}

\begin{tcolorbox}[colback=green!5!white,colframe=green!75!black,title=Tip]
Complete the
\href{https://brilliant.org/courses/molecular-representations/}{Molecules course
on Brilliant} as it is one of the best resources out there to quickly ease into
the domain.
\end{tcolorbox}

\begin{tcolorbox}[colback=red!5!white,colframe=red!75!black,title=Warning] I
wrote this as a refresher to my future that I should be able to read in an hour
or so to recall previous knowledge. My idiolect may not parse well in your
brain so buyer beware!
\end{tcolorbox}

\subsection{Atom Definition} Atoms\index{atom} are considered the smallest units
of ordinary matter with the properties of a chemical element and are defined in
terms of their constituting parts - neutrons, protons and electrons.

\subsection{Atom Composition} Atoms are composed of a nucleus\index{nucleus} and
negatively charged electrons\index{electron}. Electrons are bound to the nucleus
by the electromagnetic force.  Nucleons\index{nucleon}, which are either
positively charged protons\index{proton} or neutral neutrons\index{neutron},
make up the nucleus of an atom.

\subsection{Atom Properties} An atom's composition will determine among other
things, the charge, weight and type and isotope of an atom. Variations in the
proton count render an atom a different element. Variations in the neutron
count of an atom will result to different isotopes\index{isotope}. Variations in
the electron count produce ions\index{ion}.

\begin{table}[h] \caption{Effect of Number Change on Atom}
\begin{tabularx}{\linewidth}{|X|X|} \hline Count Change & Result \\ \hline
Proton & Element \\ Neutron & Isotope \\ Electron & Ion \\ \hline \end{tabularx}
\end{table}


\subsubsection{Element (Type)} For a given element, all atoms will contain the same
number of protons in its nucleus.  This number of protons in an atom - commonly
referred to as the atomic number \index{atomic number} $Z$ - uniquely identifies
the element of the atom or in layperson terms, the type of the material, and
forms the basis for the organization of elements as familiarized through the
periodic table of elements.

Carbon is the element defined by six protons in its nucleus and is represented
as \ce{_6}C. Since the atomic number is unique to the element one may omit the
atomic number in notation and represent carbon simply as \ce{}C without
impairing clarity, however; in this text the convention is held to present the
atomic number of an element regardless how redundant\footnote{Representing the
atomic number along with the element eliminates the need to lookup an element's
atomic number when not acquainted with a particular element}.

\subsubsection{Isotope} The number of protons for a given element is fixed,
however; the number of neutrons found within a nucleus i.e.\ neutron number $N$
of an element may vary depending on the isotope of the element.

The atomic mass number\index{atomic mass number} $A$, also known as the mass
number or \index{nucleon number}, represents the number of nucleons in an atom
and is essentially the sum of neutrons $N$ and protons $Z$.

$$A = N + Z$$

Since the proton count is fixed for an element, isotopes for a given element
require variations in neutron count in order to affect the nucleon number.

\begin{tcolorbox}[colback=blue!5!white,colframe=blue!75!black]
Since the mass of an electron is approximately 1/1836 of the mass of a proton or
neutron, we often disregarded it when calculating atomic mass. For the sake of
simplicity, we therefore calculate the atomic mass number $A$ as the sum of the
number of protons $Z$ and neutrons $N$ in an atom. Observe the following table
for more precise values for the masses of the atomic components:
\begin{tabularx}{\linewidth}{|X|X|}
\hline Component & Mass (kg) \\ \hline
Proton & $1.6726219 \times 10^{-27}$ \\
Neutron & $1.674927471 \times 10^{-27}$ \\
Electron & $9.10938356 \times 10^{-31}$ \\
\hline
\end{tabularx}
\end{tcolorbox}

Carbon's most common isotope -  carbon-12 (\ce{^{12}_{6}}C) - has an atomic mass
number of 12 ($A = 12$) with an atomic number of 6 ($Z = 6$) leading to the
conclusion that the carbon-12 isotope contains $N=A-Z=12-6=6$ neutrons.

In the case of carbon, 3 isotopes \ce{^{12}_{6}}C, \ce{^{13}_{6}}C and
\ce{^{14}_{6}}C are found naturally with the atomic number of 6 and the
respective atomic mass numbers of 12, 13 and 14.

\subsubsection{Ions (Charge)}

Electrons and protons determine the electric charge of an atom. Electrons
exhibit a negative charge \SI{-1}{\elementarycharge}, while protons exhibit a
positive charge \SI{1}{\elementarycharge}.

A neutral atom contains an equal number of electrons to protons, while an ion
contains an imbalance between electrons to protons resulting to non-zero charge.
In the case of a net positive charge, an atom is referred to as a
cation\index{cation}, while an atom with a net negative charge is referred to as
an anion\index{anion}.

Creation of an ion is accomplished through chemical or physical means. Through
chemical means, ions can be produced through acid-based reactions, dissolution,
ionization (of gases) and redox reactions. Physical means of producing ions
include electrolysis, ionizing radiation, heat and high-energy collisions.

\subsection{Behavior} Atoms, typically sized around \SI{100}{\pico\meter}, are
so small that classical physics fails to accurately describe their behavior due
to quantum effects.

\subsection{Mass and Weight} The colloquial usage of the term ``weight'' is
technically incorrect as the intent is actually to use the term ``mass''.

In physics mass is measured in grams (\si{\gram}) while weight is measured in
Newtons (\si{\newton}).  Weight\index{weight} $W = m\cdot g$ represents the
force on a mass due to gravity, as such weight is very dependent on the
gravitational pull that a mass is subjected to.  The accelleration of an object
due to its subjection to a gravitational pull is referred to as
gravity\index{gravity} $g = a_g = \frac{v}{t} = \frac{d}{t^2}$ which, on the
surface of Earth, results to an earthbound acceleration of about
\SI{9.8}{\meter\second^{-2}}.

In chemistry mass is expressed in atomic mass units (\si{\atomicmassunit}) or
Daltons (\si{\dalton}).

\begin{equation}\label{eq:amu-def} \mathrm {u} = m_{\mathrm {u} }=\frac {1}{12}
m_a(\mathrm{ \ce{^{12}}C }) \end{equation}

Atomic mass units\index{atomic mass unit} have been standardized to the unified
atomic mass unit \si{\atomicmassunit} in 1961 to refer to the mass of
\ce{^{12}}C (see equation \ref{eq:amu-def}) in order to address the complication
and confusion that arose out of chemists and physicists upholding different
standards for atomic mass units (amu) prior.

\begin{tcolorbox}[colback=gray!5!white,colframe=gray!75!black]
Within Chemistry naturally occurring oxygen (including the heavier \ce{^{17}}O and \ce{^{18}}O) were
considered as a reference for 16 atomic mass units whereas in Physics the pure
isotopic \ce{^{16}}O was considered the reference for the same 16 atomic mass
units, accounting for a difference of \num{1.000282} at times between the two
fields.
\end{tcolorbox}

Quarks\index{quark} are elementary particles that combine to form composite
particles called hadrons\index{hadron}, of which protons and neutrons are the
most stable.  Hadrons are held together by the \index{strong force} and can be
classified into baryons\index{baryon}, which are composed of 3 quarks and
mesons\index{meson} which are composed of one quark and antiquark.

Protons and neutrons are baryons and therefore participate in the strong
interaction.

The mass of an atom is largely determined by baryons as is most of the mass in
the visible universe. The rest mass of a quark only contributes a small fraction
of the mass of a proton while the quantum chromodynamics binding energy, which
includes the kinetic energy of the quarks and the energy of the gluon fields that bind the quarks together, determines the remaining bulk of the mass.

Both protons and neutrons have a mass close to a single atomic mass unit.

%Electrons, being leptons of the charged variety, do not undergo strong
%interactions

Since atoms operate at a quantum level, they are subjected to dynamics that are
predominantly dictated by forces other than gravitational force.  Postulating
the notion of ``weight'' in a classical mechanical sense is somewhat of a futile
exercise, therefore relative atomic weight is a unitless quantity defined by the
weighted average of the atomic mass of the different isotopes for an element, by
abundance.

For a given isotope $i$ of element $\mathrm E$ in material $P$, one can
calculate the atomic weight in reference to the atomic mass of \ce{^{12}_{6}}C
as represented by equation \ref{eq:atomicweightforisotope}.

\begin{equation}\label{eq:atomicweightforisotope} A_r(\mathrm { \ce{^i}E })_P =
\frac{ m_a(\mathrm { \ce{^i}E })_P }{ m_a(\mathrm { \ce{^{12}}C })/12 } = \frac{
m_a(\mathrm { \ce{^i}E })_P }{ \si{\dalton} } \end{equation}

Determining the atomic weight for element $\mathrm E$ in material $P$,
$A_r(\mathrm E)_P$, demands the factoring of the isotopic abundance $x(\mathrm {
\ce{^{i}E} })_P$ and the atomic weight of the given isotope $A_r(\mathrm {
\ce{^i}E })_P$.

\begin{equation}\label{eq:atomicweightforelement} A_r(\mathrm E)_P = \sum
[x(\mathrm { \ce{^i}E })_P \times A_r( \mathrm { \ce{^i}E } )_P] \end{equation}

The standard atomic weight of carbon is \num{12.011}\cite{atomic-weights-2013},
a value that is closer to \num{12} than \num{13} or \num{14} suggesting that
\ce{^{12}_{6}}C is the more abundant or common isotope of carbon found in normal
terrestrial materials. In fact, carbon 12 accounts for 99\% of carbon on Earth.
Carbon-13 only represents 1\% of carbon while carbon-14 is only encountered in
trace amounts at a ratio of 1 to 1.5 per \num{1e12} of carbon atoms in the
atmosphere.

% \subsection{Models}
